\documentclass{article}
\usepackage{amsmath,amsthm,parskip,amssymb}
\usepackage[utf8]{inputenc}
\usepackage{mathtools}
\usepackage{graphicx}
\usepackage{rotating}
%\pagenumbering{gobble}
\begin{document}
\begin{center}
\textbf{Midterm I Solution Sketches for Pages 2-4}\footnote{I just made these to help me grade. I made some arithmetic errors somewhere, and I can't remember if I fixed them or not. :) :(}
\end{center}

1. We are given $$A=\left[\begin{array}{cc}
1 & 2\\
-1 & 4\\
3 & 1
\end{array}\right], B=\left[\begin{array}{cc}
1 & 3\\
-1 & -1\\
-5 & 4
\end{array}\right], \text{ and } C=\left[\begin{array}{cc}
1 & 2\\
4 & -8\\
\end{array}\right].$$



a.) $$2A-B=\left[\begin{array}{cc}
2-1 & 4-3\\
-2+1 & 8+1\\
6+5 & 2-4
\end{array}\right].$$


b.) $$C^{-1}=\frac{1}{-8-8}\left[\begin{array}{cc}
-8 & 2\\
-4 & 1\\
\end{array}\right].$$



c.) $AC$ will have dimension $3\times 2$, 
$$AC=\left[\begin{array}{cc}
1+8 & 2-16\\
-1+16 & -2-32\\
3+4 & 6 -8 
\end{array}\right]=\left[\begin{array}{cc}
9 & -14\\
15 & -34\\
7 & -2
\end{array}\right].$$


\bigskip
2. 

a.) det($D$) = -1(-5-0)+1(-1-0) = 5-1=4


b.) $$\text{rref}(D)=\left[\begin{array}{ccc}
1 & 0 & 0\\
0 & 1 & 0 \\
0 & 0 & 1
\end{array}\right].$$

c.) The above shows this matrix is nonsingular. Therefore, only the trivial solution exists. That is, the solution set is $\{\mathbf{0}\}$. (Corollary 3.1 in the textbook)


d.) $$A^Tx =\left[\begin{array}{ccc}
1 & -1 & 3\\
2 & 4 & 1
\end{array}\right] \left(\begin{array}{c}
x_1\\
x_2\\
x_3
\end{array}\right) = \mathbf{0}$$

We know that $x$ must have dimension $3\times 1$ to be a vector and give a valid multiplication. 

This give two equations with three unknowns;

$$x_1 -x_2 +x_3=0,$$
$$2x_1+4x_2+x_3=0.$$

The solution set is $\{x\in\mathbb{R}^3 \mid x_2=\frac{-5}{13}x_1 \text{ and }x_3=\frac{-6}{13}x_1\}.$

\bigskip


3. 

a.) $Ax=0$ blah blah

b.) A square matrix is singular if the determinant is zero. For these matrices, no inverse exists, so that if $S$ is singular, then there does not exist a matrix $S^{-1}$ such that $SS^{-1}=I$. For example, $$S=\left[\begin{array}{c}
0 
\end{array}\right].$$

c.) Two matrices are row equivalent if one is the product of a sequence of elementary row operations and the other. 

Elementary row operations are row swaps, addition of rows, and scaling of a row by a constant. 

$[1]$ and $[2]$ are row equivalent. But perhaps you would like a less trivial example: 


$$A=\left[\begin{array}{cc}
1 & 0\\
0 & 1\\
\end{array}\right], B=\hat{E}A=\left[\begin{array}{cc}
1 & 1\\
1 & 0\\
\end{array}\right].$$

$B$ is obtained from $A$ by adding row 1 to row 2 and swapping the resulting row 2 with row 1. You might also note that $D$ and $I$ from the previous question are row equivalent, as $I$=rref($D$). 










\end{document}
