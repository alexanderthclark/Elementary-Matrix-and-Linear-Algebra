\documentclass{article}
\usepackage{amsmath,amsthm,parskip,amssymb}
\usepackage[utf8]{inputenc}
\usepackage{mathtools}
\usepackage{graphicx}
\usepackage{rotating}
\pagenumbering{gobble}
\begin{document}

Name:\\
\medskip
Section (time):

\subsection*{Math 340 Quiz 7}

1.) What is the span of $\{1-x, 1+x\}$ in the set of polynomials of degree $n$?

2.) Is the following set of vectors linearly independent? Show why or why not.

$$\left\{ \left(\begin{array}{c}
91\\
0\\
0 \end{array}\right), \left(\begin{array}{c}
91\\
10\\
73 \end{array}\right)\right\}$$



\pagebreak

\textbf{Not Quiz 7}


A \emph{basis} $\beta$ for a set $V$ has properties

\begin{enumerate}
\item span($\beta$)=$V$
\item $\beta$ is linearly independent.
\end{enumerate}

a.) Can a collection of $n+1$ vectors be a basis for $\mathbb{R}^n$?

\bigskip{}
b.) Show a set of vectors containing $\mathbf{0}$ is linearly dependent.

\bigskip{}

c.) Show any subspace is a linearly dependent set of vectors.


\bigskip{}

d.) Show $E=\{(1,0,0), (0,1,0), (0,0,1), (1,1,1)\}$ is linearly dependent, but that any set of three of these vectors is linearly independent.

\bigskip{}

e.) (Proofy) Show if $\beta$ contains $n$ linearly independent vectors, then it must be a basis for $\mathbb{R}^n$.

\bigskip{}

f.) Find a basis for the subspace $\mathcal{U}=\{(x,y,z)\in \mathbb{R}^3 \mid x-z=0\}.$

\bigskip{}
HW28.) Find a basis for $\mathbb{R}^3$ that includes

i.) $\left(\begin{array}{c}
1\\
0\\
2 \end{array}\right)$

ii.) $\left(\begin{array}{c}
1\\
0\\
2 \end{array}\right)$ and $\left(\begin{array}{c}
0\\
1\\
3 \end{array}\right)$




\pagebreak

\textbf{Not Quiz 7 Solutions}

\it Preamble: Questions a.) and e.) are difficult to show formally, and I didn't quite forecast this appropriately as I constructed the handout. So, let's not worry about proving these things. Accordingly, I have not typed up full proofs. I put them on the handout because I think they are important and helpful properties to have a feel for. You should spend some time convincing yourself that the answer to a.) is no and that the statement in e.) is true. 
\rm


a.) No. They would not be linearly independent.\footnote{Proving linear dependence takes a little work. I used induction.}


b.) To prove linear dependence of vectors $u_1, \dots, u_n$, you need only show that there is a nontrivial solution to $\sum_{i=1}^n a_i u_i =0$. Let $u_1=\mathbf{0}$, $a_1=91$, and $a_i=0$ for all $i>1$. That's a nontrivial solution, so the set of vectors is linearly dependent. 


c.) A subspace will contain a zero element $\mathbf{0}$. By part b.), it will be linearly dependent. 

d.) Observe we can subtract $(1,1,1)$ from the sum of the standard basis vectors to obtain $\mathbf{0}$. Therefore, the entire set is linearly dependent. 

There are four combinations to consider. Each will produce a nonsingular matrix. For those combinations including $(1,1,1)$, elementary row operations will eventually allow you to create an identity matrix. For those combinations not including $(1,1,1)$, the identity matrix is immediately formed by properly ordering the vectors. If the associated matrix is nonsingular, the vectors are linearly independent.


e.) See Theorem 4.12 for the whole proof. Just have some idea that $n$ vectors can span an $n$-dimensional space.


f.) The condition for a vector belonging to the set $x=z$. So, we must construct one vector that features $x=z$ with $z$ and $x$ nonzero. Let's choose $(1,0,1)$. There are no restrictions on $y$, so a vector like $(0,1,0)$, when scaled and added to $(1,0,1)$, let's us achieve any $y$ value we'd like. 

Together, this forms a basis $\{(1,0,1), (0,1,0)\}$, though many more exist.




\end{document}


have a upper tri submatrix for cofactor ease
Find det of .5*swap*