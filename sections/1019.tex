\documentclass{article}
\usepackage{amsmath,amsthm,parskip,amssymb}
\usepackage[utf8]{inputenc}
\usepackage{mathtools}
\usepackage{graphicx}
\usepackage{rotating}
\pagenumbering{gobble}
\begin{document}

Name:\\
\medskip
Section (time):

\subsection*{Math 340 Quiz 6}

1.) Give an example of two vectors in $\mathbb{R}^2$ that are equal and have different tails. 

2.) Let $\vec{PQ}$ have tail $(1,0)$ and head $(2,1)$.

Let $\vec{PS}$ have tail $(1,0)$ and head $(2,-1)$.

Draw these vectors in a graph and argue they are not equal.


\pagebreak

\textbf{Not Quiz 6}

a.) Let $\mathbb{R}_{++}$ be the set of all strictly positive real numbers and define, for $x,y\in \mathbb{R}_{++}$ a vector sum by $$x \oplus y = x \cdot y,$$

where the product on the right is the usual product of numbers. If $a$ is a real number and $x\in\mathbb{R}_{++}$ define

$$a \odot x = x^a.$$

Is this a vector space?

b.) Which of the following are subspaces?

\begin{enumerate}
\item $\{ (x_1, x_2, x_3)\in \mathbb{R}^3 \mid x_1 = 0\}$
\item $\{ (x_1, x_2, x_3)\in \mathbb{R}^3 \mid x_2 = 0\}$
\item $\{ (x_1, x_2, x_3)\in \mathbb{R}^3 \mid x_1+x_2 = 0\}$
\item $\{ (x_1, x_2, x_3)\in \mathbb{R}^3 \mid x_1+x_2 = 1\}$
\item $\{ (x_1, x_2)\in \mathbb{R}^2 \mid x_1 \geq 0\}$




\end{enumerate}

c.) Consider $\mathcal{U}\in \mathbb{R}^n$, the subset consisting of all vectors $a$ with the property $a_1+a_2+\dots a_n=0$. Define $\oplus$ and $\cdot$ as the usual vector addition and scalar multiplication:

$$a\oplus b = (a_1+b_1, \dots, a_n + b_n),$$

$$c\odot a = (ca_1, \dots ca_n).$$

Show this is a subspace of $\mathbb{R}^n$. 

d.) (4.3 HW37) Which of the following points are on the line

$$x = 4-2t$$
$$y = -3+2t$$
$$z=4-5t$$

\begin{enumerate}
\item[i] (0,1,-6)
\item[ii] (1,2,3)
\item[iii] (4,-3,4)
\item[iv] (0,1,-1)
\end{enumerate}

e.) What is the span of $\{1-x, 1+x\}$ in the set of polynomials of degree one?



\pagebreak
\textbf{Solution sketches to Not Quiz 6}

a.) Yes. 
\begin{enumerate}
\item $\oplus$ is commutative because the normal multiplication $\cdot$ is. 
\item $\oplus$ is associative because multiplication $\cdot$ is. 
\item Here the additive identity is $1$, $x\oplus 1=x$.
\item The additive inverse of $x$ is $\frac{1}{x}$. $x\oplus 1/x=1=$the additive identity. 

\item $c\odot (x\oplus y) = (x\dot y)^c = x^c y^c = c\odot x \oplus c\odot y$

\item $(c+d)\odot x = x^{c+d} = x^c \cdot x^d = c\odot x \oplus d\odot x$

\item $c\odot (d\odot x) = {(x^d)}^c = x^{cd} = (cd)\odot x$

\item $1\odot x = x^1 = x$.
\end{enumerate}
We've now verified the eight properties of a vector space.

\bigskip

b.) We check closure of addition and multiplication by a scalar. We assume the normal addition and multiplication operations. 
\begin{enumerate}
\item yes
\item yes
\item Yes. Observe $c(x_1+x_2)=c0=0$, so $c\vec{x}$ is in the proposed set for any $c\in\mathbb{R}$. If we add two vectors, $x$ and $y$, then $(x_1+y+1)+(x_2+y_2)=(x_1+x_2)+(y_1+y_2)=0+0$, so the set is closed under addition. 
\item No, this is not closed under scalar multiplication. 
\item No, the set is not closed under scalar multiplication (of a negative). 
\end{enumerate}

\bigskip

c.) As before, we check closure under addition and scalar multiplication. 

\begin{enumerate}
\item Let $u=a\oplus b = (a_1+b_1, \dots, a_n + b_n)$ Then $u_i=a_i+b_i$ for $i=1,2,\dots,n$. Now we check if $\sum_{i=1}^n u_i = 0$. Observe $\sum_{i=1}^n u_i = \sum_{i=1}^n a_i+b_i =\sum_{i=1}^n a_i +\sum_{i=1}^n b_i = 0 + 0$. Thus, this space is closed under addition.
\item For $c\odot a$, We have to check if $\sum_{i=1}^n ca_i$ is equal to zero. $\sum_{i=1}^n ca_i = c\sum_{i=1}^n a_i = c0=0$. So, the space is closed under multiplication. 
\end{enumerate}

We now know that this is a subspace.

\bigskip
d.) HW

\bigskip
e.) The span is the set of all polynomials of degree one. Observe that the coefficient vectors are $\left( \begin{array}{c}
-1\\
1
\end{array}\right)$ and $\left( \begin{array}{c}
1\\
1
\end{array}\right)$. These are linearly independent, so they would span the entire space of $\mathbb{R}^2$. This means we can pick any coefficient vector $\left( \begin{array}{c}
\alpha \\
\beta
\end{array}\right)$ in $\mathbb{R}^2$ we want and and produce it as a linear combination of $\left( \begin{array}{c}
-1\\
1
\end{array}\right)$ and $\left( \begin{array}{c}
1\\
1
\end{array}\right)$. This corresponds to saying we can pick any polynomial $\alpha x +\beta$ and represent it as a linear combination of $1-x$ and $1+x$.

\end{document}


have a upper tri submatrix for cofactor ease
Find det of .5*swap*