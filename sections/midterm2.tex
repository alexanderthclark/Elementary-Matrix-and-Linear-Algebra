\documentclass{article}
\usepackage{amsmath,amsthm,parskip,amssymb}
\usepackage[utf8]{inputenc}
\usepackage{mathtools}
\usepackage{graphicx}
\usepackage{rotating}
%\pagenumbering{gobble}
\begin{document}
\begin{center}
\textbf{Midterm II Solution Sketches for Pages 4 and 5}\footnote{I just made these to help me grade. Don't expect anything polished.}
\end{center}

4.) 

(a) The rank of a matrix is the number of LI rows or columns. 

(b) set of linear combos

(c) number of vectors in a basis

(d) Subset of vector space that is closed under the addition and scalar multiplication operations. 


(e) 


5.) Rank + Nullity = No. of Columns

6.) 

a.) We Want $\langle x+k, x^2\rangle=0$.

$$\int_{-1}^1 (x+k)x^2dx = \int_{-1}^1 x^3+kx^2dx$$

$$=\frac{1}{4}x^4 + \frac{k}{3}x^3\bigg\rvert_{-1}^1$$

$$=(\frac{1}{4}+\frac{k}{3})-(\frac{1}{4}-\frac{k}{3})$$

$$=\frac{2k}{3}$$

$$\implies k=0.$$

%$x \times x^2$ is an odd function, so it will integrate to zero on $[-1,1]$

b.) We know $x^2$ is parallel to $\alpha x^2$ for any scalar $\alpha\in\mathbb{R}$. So, we simply solve for $\alpha$ so that $\langle \alpha x^2, \alpha x^2 \rangle =1$.

$$\int_{-1}^{1} \alpha^2 x^4 dx = \alpha^2 \frac{1}{5}x^5\bigg\rvert_{-1}^1$$

$$=\frac{1}{5}(\alpha^2 +\alpha^2)$$

So we solve $\frac{2}{5}\alpha^2 = 1$. 

$$\alpha^2 = \frac{5}{2} \implies \alpha = \pm \frac{\sqrt{5}}{\sqrt{2}}.$$

You can also try to find a polynomial so that $\cos \theta = \pm 1$. 


c.) This question asks if the angle is obtuse. An angle is obtuse when the inner product is negative. So we check $\langle x+1, x^2 \rangle$. 

Recall that $\langle x, x^2 \rangle=0$. Using the properties of inner products,

$$\langle x+1, x^2 \rangle = \underbrace{\langle x, x^2 \rangle}_{\mathclap{=0}} + \langle 1, x^2 \rangle$$

So we only have to check $\int_{-1}^1 x^2 dx$. Because $x^2\geq 0$, $\int x^2$ is always positive over a nondegenerate interval. Therefore, the angle is acute. Therefore, the angle is less than $\frac{\pi}{2}$. 

\end{document}
\left[\begin{array}{cc}
1 & 2\\
-1 & 4\\
3 & 1
\end{array}\right]