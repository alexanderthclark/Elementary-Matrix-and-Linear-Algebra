\documentclass{article}
\usepackage{amsmath,amsthm,parskip,amssymb}
\usepackage[utf8]{inputenc}
\usepackage{mathtools}
\usepackage{graphicx}
\usepackage{rotating}
\pagenumbering{gobble}
\begin{document}

Name:\\
\medskip
Section (time):

\subsection*{Math 340 Quiz 2}


1.) Solve the following system of equations by writing it in the form $Ax=b$ and solve using matrix multiplication.

$$2x+y=4$$
$$x+2y=5$$

2.) Suppose there is a function $f:\mathbb{R}^n \rightarrow \mathbb{R}^2$ defined by $f(x)=Ax$.  

We know $Au=\left(\begin{array}{c}
1 \\
1 \\
\end{array}\right)$ and $Av=\left(\begin{array}{c}
-1 \\
-1 \\
\end{array}\right)$. 

Show that $\left(\begin{array}{c}
0 \\
0 \\
\end{array}\right)$ is in the image of the function $f(x)=Ax$. 

\pagebreak

\textbf{Row operations and reduced row echelon form.}

\bigskip{}

1.) Assuming invertibility, show $(ABCD)^{-1}=D^{-1}C^{-1}B^{-1}A^{-1}$
\smallskip

2.) True or false? For any elementary matrix $E$, $E=E^{-1}$.

\smallskip

3.) Which of these matrices are in reduced row echelon form? Reduce the matrices not already in reduced form.
\smallskip

$A= \left[\begin{array}{cccc}
1 & 0 & 3 & 1 \\
0 & 0 & 1 & 0 \\
0 & 0 & 0 & 0
\end{array}\right]$ $B= \left[\begin{array}{cccccc}
1 & 1 & 1 & 1 & 1 & 1\\
0 & 0 & 1 & 0 & 0 & 0\\
0 & 0 & 0 & 1 & 2 & 3\\
0 & 0 & 0 & 0 & 1 & 1\\
0 & 0 & 0 & 0 & 0 & 1
\end{array}\right]$ 

$C= \left[\begin{array}{ccccc}
0 & 1 & 0 & 3 & 4\\
0 & 0 & 1 & 5 & 6
\end{array}\right]$ $D= \left[\begin{array}{cccccc}
1 & 0 & 1 & 0 & 1 & 0\\
0 & 1 & 1 & 0 & 2 & 1\\
0 & 0 & 0 & 1 & 3 & 0\\
0 & 0 & 0 & 0 & 0 & 0\\
\end{array}\right]$



\bigskip{}

4.) Let $A$ be a square matrix. Show that if a sequence of elementary row operations that when applied successively to $A$ yield the identity matrix, then the same operations applied in the same order to $I$ yield $A^{-1}$.

5.) Show that if a matrix $A$ is row equivalent to an inverse matrix, then it must be nonsingular.
\pagebreak

\textbf{Solutions}

\medskip{}

1.) Assuming invertibility, show $(ABCD)^{-1}=D^{-1}C^{-1}B^{-1}A^{-1}$
\smallskip

\it
We say $Y=X^{-1}$ if $XY=I$. 

We can check the product, $ABCD(D^{-1}C^{-1}B^{-1}A^{-1})$. This will collapse from the inside.

$$ABCD(D^{-1}C^{-1}B^{-1}A^{-1}) = ABCIC^{-1}B^{-1}A^{-1} = \cdots = I.$$
Note this also works if we reverse the order of multiplication. 
$$(D^{-1}C^{-1}B^{-1}A^{-1}) ABCD = D^{-1}C^{-1}B^{-1}IBCD = \cdots = I.$$


Therefore, $D^{-1}C^{-1}B^{-1}A^{-1}$ must be the inverse of $ABCD$. We can also show this by applying the rule $(XY)^{-1}=Y^{-1}X^{-1}$ twice. Let $X=AB$ and $Y=CD$. We obtain an inverse $(CD)^{-1}(AB)^{-1}=D^{-1}C^{-1}B^{-1}A^{-1}$.
\rm

2.) True or false? For any elementary matrix $E$, $E=E^{-1}$.

\it
False. Swap matrices can be their own inverse, but this isn't true for the matrices we use to represent multiplying a row by a constant. 
\rm

\smallskip

3.) Which of these matrices are in reduced row echelon form? Reduce the matrices not already in reduced form.
\smallskip

$A= \left[\begin{array}{cccc}
1 & 0 & 3 & 1 \\
0 & 0 & 1 & 0 \\
0 & 0 & 0 & 0
\end{array}\right]$ $B= \left[\begin{array}{cccccc}
1 & 1 & 1 & 1 & 1 & 1\\
0 & 0 & 1 & 0 & 0 & 0\\
0 & 0 & 0 & 1 & 2 & 3\\
0 & 0 & 0 & 0 & 1 & 1\\
0 & 0 & 0 & 0 & 0 & 1
\end{array}\right]$ 

$C= \left[\begin{array}{ccccc}
0 & 1 & 0 & 3 & 4\\
0 & 0 & 1 & 5 & 6
\end{array}\right]$ $D= \left[\begin{array}{cccccc}
1 & 0 & 1 & 0 & 1 & 0\\
0 & 1 & 1 & 0 & 2 & 1\\
0 & 0 & 0 & 1 & 3 & 0\\
0 & 0 & 0 & 0 & 0 & 0\\
\end{array}\right]$



\bigskip{}

\it $C$ and $D$ are already reduced. $A$ and $B$ must be reduced further.

$$A_{r_1-3r_2\rightarrow r_1}= \left[\begin{array}{cccc}
1 & 0 & 0 & 1 \\
0 & 0 & 1 & 0 \\
0 & 0 & 0 & 0
\end{array}\right]$$

We need several steps for $B$. First we replace row 1 with row 1 minus row 2, $BB_{r_1-r_2\rightarrow r_1}$.
$$W = B_{r_1-r_2\rightarrow r_1}= \left[\begin{array}{cccccc}
1 & 1 & 0 & 1 & 1 & 1\\
0 & 0 & 1 & 0 & 0 & 0\\
0 & 0 & 0 & 1 & 2 & 3\\
0 & 0 & 0 & 0 & 1 & 1\\
0 & 0 & 0 & 0 & 0 & 1
\end{array}\right]$$ 
Then we substract row 3 from row 1 to eliminate the nonzero above a leading one in the fourth column. 

$$X = W_{r_1-r_3\rightarrow r_1} = \left[\begin{array}{cccccc}
1 & 1 & 0 & 0 & -1 & -2\\
0 & 0 & 1 & 0 & 0 & 0\\
0 & 0 & 0 & 1 & 2 & 3\\
0 & 0 & 0 & 0 & 1 & 1\\
0 & 0 & 0 & 0 & 0 & 1
\end{array}\right]$$ 

Let's get rid of the 2, entry $(3,5)$, by substracting 2 times row 4 from frow 3

$$Y = X_{r_3-2r_4\rightarrow r_3} = \left[\begin{array}{cccccc}
1 & 1 & 0 & 0 & -1 & -2\\
0 & 0 & 1 & 0 & 0 & 0\\
0 & 0 & 0 & 1 & 0 & 1\\
0 & 0 & 0 & 0 & 1 & 1\\
0 & 0 & 0 & 0 & 0 & 1
\end{array}\right]$$ 

$$Z = Y_{r_1+r_4\rightarrow r_1} = \left[\begin{array}{cccccc}
1 & 1 & 0 & 0 & 0 & -2\\
0 & 0 & 1 & 0 & 0 & 0\\
0 & 0 & 0 & 1 & 0 & 1\\
0 & 0 & 0 & 0 & 1 & 1\\
0 & 0 & 0 & 0 & 0 & 1
\end{array}\right]$$ 

Now we have to get rid of the entries in the last column $(1,6), (3,6), \text{ and, } (4,6)$. Let's combine a few steps. We can do that by adding or subtracting the last row with the others.
$$B_{rref} = Z_{r_1+2r_5\rightarrow r_1,r_3-r_5\rightarrow r_3, r_4-r_5\rightarrow r_4 } = \left[\begin{array}{cccccc}
1 & 1 & 0 & 0 & 0 & 0\\
0 & 0 & 1 & 0 & 0 & 0\\
0 & 0 & 0 & 1 & 0 & 0\\
0 & 0 & 0 & 0 & 1 & 0\\
0 & 0 & 0 & 0 & 0 & 1
\end{array}\right]$$ 

\rm

\medskip{}

4.) Let $A$ be a square matrix. Show that if a sequence of elementary row operations that when applied successively to $A$ yield the identity matrix, then the same operations applied in the same order to $I$ yield $A^{-1}$.



\it
We know $E_1 \cdots E_n A = I$. 

We want something like $E_1 \cdots E_n I = A^{-1}$.

If you wanted to work backwards, you could multiply both sides, from the right, by $A$. 

$$E_1 \cdots E_n I A= A^{-1}A = I$$

And that's exactly what we wanted. The above reduces to $E_1 \cdots E_n  A=  I$

\rm

5.) Show that if a matrix $A$ is row equivalent to an inverse matrix, then it must be nonsingular.


\it

$A=\hat{E}B^{-1}=E_1\cdots E_n B^{-1}$

We invert the right hand side. 

$\left[\hat{E}B^{-1}\right]^{-1} = BE_n^{-1}\cdots E_1^{-1}$

We know each of these inverses exist because elementary matrices are invertible. Therefore, $A^{-1}$ exists and  $A^{-1}=BE_n^{-1}\cdots E_1^{-1}.$

\end{document}

Can a matrix with a row of all zeroes have an inverse?