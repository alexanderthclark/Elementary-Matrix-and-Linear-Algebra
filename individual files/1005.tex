\documentclass{article}
\usepackage{amsmath,amsthm,parskip,amssymb}
\usepackage[utf8]{inputenc}
\usepackage{mathtools}
\usepackage{graphicx}
\usepackage{rotating}
\pagenumbering{gobble}
\begin{document}

Name:\\
\medskip
Section (time):

\subsection*{Math 340 Quiz 4}


1.) Compute the inverse of the following matrix. Show all of your work. 

$$A=\left[\begin{array}{ccc}
\frac{1}{2} & 0 &  0\\
0 & 0 & \frac{1}{2}\\
0 & \frac{1}{2} & 0
\end{array}\right]$$
\medskip


2.) 

$$B=\left[\begin{array}{ccc}
1 & 0 &  0\\
0 & 1 & 0\\
0 & 0 & 1
\end{array}\right] \text{ and } C=\left[\begin{array}{ccc}
2 & 0 &  0\\
0 & 1 & 1\\
0 & 1 & 2
\end{array}\right]$$

What is the determinant of B? What is the determinant of $C$? Explain how you can calculate the determinant of $C$ using the properties of determinants and row operations.   

\pagebreak

\textbf{Not quiz 4: Elementary Ops, Inverses, but mostly Determinants}


a.) Factor $G=\left[\begin{array}{ccc}
0 & 1 &  1\\
1 & 0 & 1\\
1& 1 & 0
\end{array}\right]$ into the product of elementary matrices.


b.) We discussed the determinant as a function that spits out a number after you feed it a matrix. What is the image (values the function will actually take) of this determinant function if the domain is all non-singular matrices?

c.) Let $A$ be a special kind of matrix, $A^2=A$. Can the determinant be zero? Can you find possible determinant value(s) for these kinds of matrices?\footnote{This is called an idempotent matrix, but that's not the point.}

\medskip

d.) Find the determinant of:

$A = \left[\begin{array}{ccc}
1 & 0 &  0\\
0 & 1 & 0\\
0 & 0 & 1
\end{array}\right] + \left[\begin{array}{ccc}
1 & 0 &  0\\
0 & 1 & 0\\
0 & 0 & 1
\end{array}\right]$

$B = \left[\begin{array}{ccc}
1 & 0 &  0\\
0 & 1 & 0\\
0 & r & 1
\end{array}\right]$ where $r\in \mathbb{R}$.


We can say that the determinant behaves like a linear function \emph{on the rows} of a matrix. Can you provide an example of this by expressing det($B$) as a sum of two determinants?

e.) Suppose $A$ is a square, $n\times n$ matrix. What is the determinant of $tA$? 


f.) Consider matrices: 

$$A = \left[\begin{array}{ccc}
1 & 0 &  0\\
0 & 1 & 0\\
0 & 0 & 1
\end{array}\right] \text{ and } B=\left[\begin{array}{ccc}
1 & 0 &  0\\
0 & 1 & 1\\
0 & 1 & 1
\end{array}\right].$$

Billy claims both have the same determinant because $B$ was obtained by adding row 2 to row 3 and row 3 to row 2. Is he right?$^{(no)}$ What is his mistake?

HW30.) Let $A$ be a $3\times 3$ matrix with det($A$)=3. What is the rref to which $A$ is row equivalent? How many solutions does the homogeneous system $Ax=0$ have?

\pagebreak{}

\textbf{Solutions}

a.) This is nonsingular so it will be row equivalent to the identity matrix. 

$G=\hat{E}I = \hat{E}=E_1\cdots E_n.$

$H=G_{r_3+r_2\rightarrow r_2} = \left[\begin{array}{ccc}
0 & 1 &  1\\
2 & 1 & 1\\
1& 1 & 0
\end{array}\right] $

$J=H_{r_1+r_3\rightarrow r_3} = \left[\begin{array}{ccc}
0 & 1 &  1\\
2 & 1 & 1\\
1& 2 & 1
\end{array}\right] $


$K=J_{(r_2-r_1)/2\rightarrow r_2} = \left[\begin{array}{ccc}
0 & 1 & 1 \\
1 & 0 & 0\\
1& 2 & 1
\end{array}\right]$

$L=K_{r_3-r_2-r_1\rightarrow r_3} = \left[\begin{array}{ccc}
0 & 1 & 1 \\
1 & 0 & 0\\
0& 1 & 0
\end{array}\right]$

$M=L_{r_1-r_3-\rightarrow r_1} = \left[\begin{array}{ccc}
0 & 0 & 1 \\
1 & 0 & 0\\
0& 1 & 0
\end{array}\right]$

$N=L_{r_1\leftrightarrow r_2 \text{ then } r_2 \leftrightarrow r_3} = \left[\begin{array}{ccc}
1 & 0 & 0 \\
0 & 1 & 0\\
0& 0 & 1
\end{array}\right]$

So the elementary matrices describe these operations. 


$G=\left[\begin{array}{ccc}
1 & 0 & 0 \\
0 & 1 & 1\\
0& 0 & 1
\end{array}\right] 
\left[\begin{array}{ccc}
1 & 0 & 0 \\
0 & 1 & 0\\
1& 0 & 1
\end{array}\right] \left[\begin{array}{ccc}
1 & 0 & 0 \\
-1/2 & 1/2 & 0\\
0& 0 & 1
\end{array}\right] \left[\begin{array}{ccc}
1 & 0 & 0 \\
0 & 1 & 0\\
-1& -1 & 1
\end{array}\right] \left[\begin{array}{ccc}
1 & 0 & -1 \\
0 & 1 & 0\\
0& 0 & 1
\end{array}\right] \left[\begin{array}{ccc}
0 & 1 & 0 \\
1 & 0 & 0\\
0& 0 & 1
\end{array}\right] \left[\begin{array}{ccc}
1 & 0 & 0 \\
0 & 0 & 1\\
0& 1 & 0
\end{array}\right]$

$G$ = (add rows)(add rows)(subtract and scale)(subtract rows)(subtract rows)(swap)(swap).

You may find a different answer that is still equivalent. 

b.) $\mathbb{R}\setminus \{0\}$, the reals except zero.

c.) The determinant must be one. The most direct way is using properties:

i.) det($AB$)=det($A$)det($B$)

ii.) det($A^{-1}$)= det($A$)$^{-1}$

Note $A=A^{-1}$. Then det($AA$) =det($A$)/det($A$) = det($I$) = 1 $\implies$ det($A$)=1.

d.) det($A$)=8

det($B$)=1

$\vert B\vert =\left\vert \begin{array}{ccc}
1 & 0 & 0 \\
0 & 1 & 0\\
0& 0 & 1
\end{array}\right\vert  + \left\vert \begin{array}{ccc}
1 & 0 & 0 \\
0 & 1 & 0\\
0& r & 0
\end{array}\right\vert = 1 + 0.$

e.) $t^n$det($A$).

f.) He added rows not in sequence. 

HW30.) It's homework.






\end{document}